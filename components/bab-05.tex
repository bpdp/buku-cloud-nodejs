\chapter{Node.js dan Web: Teknik Pengembangan Aplikasi}

\section{Pendahuluan}

Pada saat membangun aplikasi Cloud dengan antarmuka web menggunakan Node.js, ada beberapa teknik pemrograman yang bisa digunakan. Bab ini akan membahas berbagai teknik tersebut. Untuk mengerjakan beberapa latihan di bab ini, digunakan suatu file dengan format JSON. File \textit{pegawai.json} berikut ini akan digunakan dalam pembahasan selanjutnya.

\lstset{language=JavaScript,caption=pegawai.json}
\lstinputlisting{src/bab-05/pegawai.json}

\index{JSON!Validator}Jika ingin memeriksa validitas dari data berformat JSON, pemrogram bisa menggunakan validator di \url{http://jsonlint.com}.

\section{\textit{Event-Driven Programming} dan EventEmitter}

\index{Event-Driven}\textit{Event-Driven Programming} (selanjutnya akan disebut EDP) atau sering juga disebut \textit{Event-Based Programming} merupakan teknik pemrograman yang menggunakan \textit{event} atau suatu kejadian tertentu sebagai pemicu munculnya suatu aksi serta aliran program. Contoh event misalnya adalah sebagai berikut:
\begin{itemize}
	\item Menu dipilih.
	\item Tombol "Submit" di klik.
	\item Server menerima permintaan dari klien.
\end{itemize}
Pada dasarnya ada beberapa bagian yang harus disiapkan dari paradigma dan teknik pemrograman ini:
\begin{itemize}
	\item \textit{main loop} atau suatu konstruksi utama program yang menunggu dan mengirimkan sinyal event.
	\item definisi dari berbagai event yang mungkin muncul
	\item definisi \textit{event-handler} untuk menangani event yang muncul dan dikirimkan oleh \textit{main loop}
\end{itemize}
\index{events.EventEmitter}Node.js merupakan peranti pengembangan yang menggunakan teknik pemrograman ini. Pada Node.js, EDP ini semua dikendalikan oleh kelas \textit{events.EventEmitter}. Jika ingin menggunakan kelas ini, gunakan \textit{require('events')}. Dalam terminologi Node.js, jika suatu event terjadi, maka dikatakan sebagai \textit{emits an event}, sehingga kelas yang digunakan untuk menangani itu disebut dengan events.EventEmitter. Pada dasarnya banyak event yang digunakan oleh berbagai kelas lain di Node.js. Contoh kecil dari penggunaan itu diantaranya adalah \textit{net.Server} yang meng-\textit{emit} event "connection", "listening", "close", dan "error".

Untuk memahami mekanisme ini, pahami dua kode sumber berikut:
\begin{itemize}
	\item server.js: mengaktifkan server http (diambil dari manual Node.js)
	\item server-on-error.js: mencoba mengaktifkan server pada host dan port yang sama dengan server.js. Aktivasi ini akan menyebabkan Node.js meng-\textit{emit} event 'error' karena host dan port sudah digunakan di server.js.
\end{itemize}
File server.js dijalankan lebih dulu, setelah itu baru menjalankan server-on-error.js.

\lstset{language=JavaScript,caption=server.js}
\lstinputlisting{src/bab-05/edp/server.js}

\lstset{language=JavaScript,caption=server-on-error.js}
\lstinputlisting{src/bab-05/edp/server-on-error.js}

\section{Asynchronous / Non-blocking IO dan \textit{Callback}}

\index{Asynchronous}\index{Non-blocking IO}\index{Callback}\textit{Asynchronous input/output} merupakan suatu bentuk pemrosesan masukan/keluaran yang memungkinkan pemrosesan dilanjutkan tanpa menunggu proses tersebut selesai. Saat pemrosesan masukan/keluaran tersebut selesai, hasil akan diberikan ke suatu fungsi. Fungsi yang menangani hasil pemrosesan saat pemrosesan tersebut selesai disebut \textit{callback} (pemanggilan kembali). Jadi, mekanismenya adalah: proses masukan/keluaran - lanjut ke alur berikutnya - panggil kembali fungsi pemroses jika proses masukan/keluaran sudah selesai.

\lstset{language=JavaScript,caption=Membaca file secara synchronous}
\lstinputlisting{src/bab-05/synchronous.js}

\lstset{language=JavaScript,caption=Membaca file secara asynchronous}
\lstinputlisting{src/bab-05/asynchronous-callback.js}

\chapter{Mengakses Basis Data NoSQL: mongoDB}

\section{Apa itu Basis Data NoSQL?}

Pada awalnya, istilah NoSQL digunakan oleh Carlo Strozzi untuk menyebut nama software basis data yang dibuat olehnya. Software basis data tersebut tidak mengikuti standar SQL, sehingga dia menyebut software tersebut dengan "NoSQL"\footnote{\url{http://www.strozzi.it/cgi-bin/CSA/tw7/I/en_US/nosql/Home\%20Page}}. Setelah itu, istilah NoSQL dipopulerkan oleh Eric Evans untuk menyebut jenis software basis data yang tidak menggunakan standar SQL. Dalam perkembangan berikutnya, NoSQL ini lebih diarahkan pada "Not Only SQL" dan digunakan untuk kategorisasi basis data \textit{non-relational} (misalnya OODBMS, Graph Database, Document-oriented, dan lain-lain). Meski ada usaha untuk menstandarkan bahasa \textit{query} untuk NoSQL (UnQL - \textit{Unstructured Query Language}), sampai saat ini usaha tersebut tidak menghasilkan sesuatu hal yang disepakati bersama karena dunia NoSQL memang kompleks sekali. Untuk melihat daftar dari basis data NoSQL, anda bisa melihat ke \url{http://nosql-databases.org}.

\section{Mengenal mongoDB dan Fitur-fiturnya}

\section{Node.js dan MongoDB}

\section{Aplikasi Web Menggunakan Node.js dan MongoDB}



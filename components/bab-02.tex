\chapter{REPL dan Dasar-dasar JavaScript di Node.js}

\section{REPL}

\index{REPL}REPL adalah lingkungan pemrograman interaktif, tempat developer bisa mengetikkan program per baris dan langsung mengeksekusi hasilnya. Biasanya ini digunakan untuk menguji perintah-perintah yang cukup dijalankan pada satu baris atau satu blok segmen kode sumber saja. Karena fungsinya itu, maka istilah yang digunakan adalah REPL (read-eval-print-loop), yaitu loop atau perulangan baca perintah - evaluasi perintah - tampilkan hasil. REPL sering juga disebut sebagai \textit{interactive top level} atau \textit{language shell}. ``Tradisi'' ini sudah dimulai sejak jaman LISP di mesin UNIX di era awal pengembangan \textit{development tools}. Saat ini hampir semua \textit{interpreter/compiler} mempunyai REPL, misalnya Python, Ruby, Scala, PHP, berbagai interpreter/compiler LISP, dan tidak ketinggalan Node.js. 

\subsection{Mengaktifkan REPL}

Untuk mengaktifkan REPL dari Node.js, \textit{executable command line program}-nya adalah \textbf{node}. Jika \textbf{node} dipanggil dengan argumen nama file JavaScript, maka file JavaScript tersebut akan dieksekusi, sementara jika tanpa argumen, akan masuk ke REPL:

\lstset{language=bash,caption=Node.js REPL}
\lstinputlisting{src/non-nodejs/bab-02/nodejs-repl.txt}

Tanda ``\textbf{>}'' adalah tanda bahwa REPL Node.js siap untuk menerima perintah. Untuk melihat perintah-perintah REPL, bisa digunakan \textbf{.help}.

\subsection{Perintah-perintah REPL}

Pada sesi REPL, kita bisa memberikan perintah internal REPL maupun perintah-perintah lain yang sesuai dan dikenali sebagai perintah JavaScript. Perintah internal REPL Node.js terdiri atas:
\begin{itemize}
  \item \textbf{.break}: keluar dan melepaskan diri dari "keruwetan" baris perintah di REPL.
  \item \textbf{.clear}: alias untuk .break
  \item \textbf{.exit}: keluar dari sesi REPL (bisa juga dengan menggunakan Ctrl-D)
  \item \textbf{.help}: menampilkan pertolong perintah internal REPL
  \item \textbf{.load}: membaca dan mengeksekusi perintah-perintah JavaScript yang terdapat pada suatu file.
  \item \textbf{.save}: menyimpan sesi REPL ke dalam suatu file.
\end{itemize}

Contoh untuk \textbf{.load}:

\lstset{language=JavaScript,caption=Contoh penggunaan .load dalam REPL}
\lstinputlisting{src/non-nodejs/bab-02/nodejs-repl-load.txt}

Setelah keluar dari sesi REPL, maka port akan ditutup dan hasil eksekusi di atas akan dibatalkan. 

Untuk menyimpan hasil sesi REPL menggunakan \textbf{.save}, jika tanpa menyebutkan direktori, maka akan disimpan di direktori aktif saat itu. Contoh:
\lstset{language=bash,caption=Contoh penggunaan perintah .save di sesi REPL}
\lstinputlisting{src/non-nodejs/bab-02/nodejs-repl-save.txt}

\section{Dasar-dasar JavaScript di Node.js}

Node.js merupakan sistem peranti lunak yang merupakan implementasi dari bahasa pemrograman JavaScript. Spesifikasi JavaScript yang diimplementasikan merupakan spesifikasi resmi dari \index{ECMAScript}ECMAScript serta \index{CommonJS}CommonJS (\url{http://commonjs.org}). Dengan demikian, jika anda sudah pernah mempelajari JavaScript sebelumnya, tata bahasa dari perintah yang dipahami oleh Node.js masih tetap sama dengan JavaScript. 

\subsection{Membaca \textit{Masukan} dari Stream / Masukan Standar (stdin)}

Untuk lebih memahami dasar-dasar JavaScript serta penerapannya di Node.js, seringkali kita perlu melakukan simulasi pertanyaan - proses - keluaran jawaban. Proses akan kita pelajari seiring dengan materi-materi berikutnya, sementara untuk keluaran, kita bisa menggunakan \textbf{console.log}. Bagian ini akan menjelaskan sedikit tentang masukan.

\index{Readline}Perintah untuk memberi masukan di Node.js sudah tersedia pada pustaka API \textit{Readline}\footnote{Lengkapnya bisa diakses di \url{http://nodejs.org/api/readline.html}}. Pola dari masukan ini adalah sebagai berikut:
\begin{itemize}
  \item me-\textit{require} pustaka Readline
  \item membuat \textit{interface} untuk masukan dan keluaran
  \item .. gunakan interface ..
  \item .. gunakan interface ..
  \item .. gunakan interface ..
  \item .. gunakan interface ..
  \item ..
  \item ..
  \item tutup \textit{interface}
\end{itemize}

Implementasi dari pola diatas bisa dilihat pada kode sumber berikut ini (diambil dari manual Node.js):

\lstset{language=JavaScript,caption=readline.js: penggunaan pustaka Readline untuk masukan}
\lstinputlisting{src/bab-02/readline.js}

\begin{Sbox}
\begin{minipage}{\textwidth}
\textbf{Catatan:} \textit{function(answer)} pada listing di atas merupakan \textit{anonymous function} atau fungsi anonimus (sering juga disebut \textit{lambda function} / fungsi lambda. Posisi fungsi pada listing tersebut disebut dengan fungsi \textit{callback}. Untuk keperluan pembahasan saat ini, untuk sementara yang perlu dipahami adalah hasil input akan dimasukkan ke \textit{answer} untuk diproses lebih lanjut. Fungsi dan \textit{callback} akan dibahas lebih lanjut pada pembahasan berikutnya.
\end{minipage}
\end{Sbox}
\begin{center}
\shadowbox{\TheSbox}
\end{center}

\subsection{Nilai/Value dan Tipe Data}

\index{Tipe data}Program dalam JavaScript akan berhubungan dengan data atau nilai. Setiap nilai mempunyai tipe tertentu. JavaScript mengenali berbagai tipe berikut ini:
\begin{itemize}
  \item Angka: bulat (misalnya 4) atau pecahan (misalnya 3.75)
  \item \textit{Boolean}: nilai benar (true) dan salah (false)
  \item String: diapit oleh tanda petik ganda ("contoh string") atau tunggal ('contoh string')
  \item \textit{null}
  \item \textit{undefined}
\end{itemize}

\index{Dynamically typed}JavaScript adalah bahasa pemrograman yang mengijinkan pemrogram untuk tidak mendefinisikan tipe data pada saat deklarasi, atau sering juga disebut sebagai \textit{dynamically typed language}:

\lstset{language=JavaScript,caption=Fitur \textit{dynamically typed language}}
\lstinputlisting{src/bab-02/dynamic.js}

Pada contoh di atas, kita bisa melihat bahwa data akan dikonversi secara otomatis pada saat program dieksekusi.

\begin{Sbox}
\begin{minipage}{\textwidth}
\textbf{Catatan:}
\begin{itemize}
  \item Khusus untuk operator "+", JavaScript akan melakukan penggabungan string (\textit{string concatenation}), tetapi untuk operator lain, akan dilakukan operasi matematis sesuai operator tersebut (-,/,*).
  \item Konversi string ke tipe numerik bisa dilakukan dengan \textit{parseInt(string)} (jika bilangan bulat) dan \textit{parseFloat(string)} (jika bilangan pecahan).
\end{itemize}
\end{minipage}
\end{Sbox}
\begin{center}
\shadowbox{\TheSbox}
\end{center}

\subsection{Variabel}

\index{Variabel}Variabel adalah suatu nama yang didefinisikan untuk menampung suatu nilai. Nama ini akan digunakan sebagai referensi yang akan menunjukkan ke nilai yang ditampungnya. Nama variabel disebut \index{Identifier}dengan \textit{identifier} / pengenal. Ada beberapa syarat pemberian nama \textit{identifier} di JavaScript: 
\begin{itemize}
  \item Dimulai dengan huruf, \textit{underscore} (\_), atau tanda dollar (\$).
  \item Karakter berikutnya bisa berupa angka, selain ketentuan pertama di atas.
  \item Membedakan huruf besar - kecil.
\end{itemize}
Konvensi yang digunakan oleh pemrogram JavaScript terkait dengan penamaan ini adalah variasi dari metode \textit{camel case}, yaitu \textit{camelBack}. Contoh: jumlahMahasiswa, linkMenu, status.

\subsection{Konstanta}

\index{Konstanta}Konstanta mirip dengan variabel, hanya saja sifatnya \textit{read-only}, tidak bisa diubah-ubah setelah ditetapkan. Untuk menetapkan konstanta di JavaScript, digunakan kata kunci \textit{const}. Contoh: 

\lstset{language=JavaScript,caption=Contoh konstanta dalam JavaScript}
\lstinputlisting{src/bab-02/const.js}

Konvensi penamaan konstanta adalah menggunakan huruf besar semua. Bagian ini (sampai saat buku ini ditulis) hanya berlaku di Firefox dan Google Chrome - V8 (artinya berlaku juga untuk Node.js).

\subsection{Fungsi}

\subsubsection{Pengertian Fungsi}

\index{Fungsi}Fungsi merupakan subprogram atau suatu bagian dari keseluruhan program yang ditujukan untuk mengerjakan suatu pekerjaan tertentu dan (biasanya) menghasilkan suatu nilai kembalian. Subprogram ini relatif independen terhadap bagian-bagian lain sehingga memenuhi kaidah "bisa-digunakan-kembali" atau \textit{reusable} pada beberapa program yang memerlukan fungsionalitasnya. Fungsi dalam ilmu komputer sering kali juga disebut dengan \textit{prcedure, routine}, atau \textit{method}.

\subsubsection{Definisi Fungsi}

Definisi fungsi dari JavaScript di Node.js bisa dilakukan dengan sintaksis berikut ini:

\lstset{language=JavaScript,caption=Sintaksis Fungsi dalam JavaScript}
\lstinputlisting{src/non-nodejs/bab-02/sintaksis-fungsi.txt}

Setelah dideklarasikan, fungsi tersebut bisa dipanggil dengan cara sebagai berikut:

\lstset{language=JavaScript,caption=Pemanggilan Fungsi dalam JavaScript}
\lstinputlisting{src/non-nodejs/bab-02/pemanggilan-fungsi.txt}

Contoh dalam program serta pemanggilannya adalah sebagai berikut:

\lstset{language=bash,caption=Contoh deklarasi fungsi dan pemanggilannya}
\lstinputlisting{src/non-nodejs/bab-02/contoh-fungsi.txt}

\subsubsection{Fungsi Anonim}

\index{Fungsi Anonim}Fungsi anonim adalah fungsi tanpa nama, pemrogram tidak perlu memberikan nama ke fungsi. Biasanya fungsi anonimus ini hanya digunakan untuk fungsi yang dikerjakan pada suatu bagian program saja dan tidak dengan maksud untuk dijadikan komponen yang bisa dipakai di bagian lain dari program (biasanya untuk menangani \textit{event} atau \textit{callback}). Untuk mendeklarasikan fungsi ini, digunakan literal \textit{function}.

\lstset{language=JavaScript,caption=Fungsi anonim}
\lstinputlisting{src/bab-02/fungsiAnonim.js}

\subsubsection{Fungsi Rekursif}

\index{Fungsi rekursif}Fungsi rekursif adalah fungsi yang memanggil dirinya sendiri. Contoh dari aplikasi fungsi rekursif adalah pada penghitungan faktorial berikut:

\lstset{language=JavaScript,caption=Fungsi rekursif untuk menghitung faktorial}
\lstinputlisting{src/bab-02/fungsiRekursif.js}

\subsubsection{Fungsi di dalam Fungsi / \textit{Nested Functions}}

\index{Nested functions}Saat mendefinisikan fungsi, di dalam fungsi tersebut pemrogram bisa mendefinisikan fungsi lainnya. Meskipun demikian, fungsi yang terletak dalam suatu definisi fungsi tidak bisa diakses dari luar fungsi tersebut dan hanya tersedia untuk fungsi yang didefinisikan.

\lstset{language=JavaScript,caption=Fungsi di dalam Fungsi}
\lstinputlisting{src/bab-02/nested.js}

\subsection{Literal}

\index{Literal}Literal digunakan untuk merepresentasikan nilai dalam JavaScript. Ada beberapa tipe literal.

\subsubsection{Literal Array}

\index{Array}Array atau variabel berindeks adalah penampung untuk obyek yang menyerupai \textit{list} atau daftar. Obyek array juga menyediakan berbagai fungsi dan metode untuk mengolah anggota yang terdapat dalam daftar tersebut (terutama untuk operasi \textit{traversal} dan permutasi. Listing berikut menunjukkan beberapa operasi untuk literal array.

\lstset{language=JavaScript,caption=Array di JavaScript}
\lstinputlisting{src/bab-02/array.js}

\subsubsection{Literal Boolean}

\index{Boolean}Literal boolean menunjukkan nilai benar (true) atau salah (false).

\subsubsection{Literal Integer}

\index{Integer}Literal integer digunakan untuk mengekspresikan nilai bilangan bulat. Nilai bulangan bulat dalam JavaScript bisa dalam bentuk:
\begin{itemize}
  \item decimal (basis 10): digit tanpa awalan nol.
  \item octal (basis 8): digit diawali dengan 1 angka nol.\footnote{pada ECMA-262, bilangan octal ini sudah tidak digunakan lagi.}
  \item hexadecimal (basis 16): digit diawali dengan 0x.
\end{itemize}

\subsubsection{Literal Floating-point}

\index{Floating-point}Literal ini digunakan untuk mengekspresikan nilai bilangan pecahan, misalnya 0.4343 atau bisa juga menggunakan E/e (nilai eksponensial), misalnya 
-3.1E12.

\subsubsection{Literal Obyek}

\index{Obyek}Literal ini akan dibahas di bab yang menjelaskan tentang paradigma pemrograman berorientasi obyek di JavaScript.

\subsubsection{Literal String}

\index{String}Literal string mengekspresikan suatu nilai dalam bentuk sederetan karakter dan berada dalam tanda petik (ganda/``'' maupun tunggal/''). Contoh:
\begin{itemize}
   \item ``Kembali ke halaman utama''
   \item 'Lisensi'
   \item ``Hari ini, Jum'at, tanggal 21 November''
   \item ``1234.543''
   \item ``baris pertama \verb+\n+ baris kedua''
 \end{itemize}

Contoh terakhir di atas menggunakan karakter khusus (\verb+\n+). Beberapa karakter khusus lainnya adalah:

\begin{itemize}
  \item \verb+\b+: Backspace
  \item \verb+\f+: Form feed
  \item \verb+\n+: New line
  \item \verb+\r+: Carriage return
  \item \verb+\t+: Tab
  \item \verb+\v+: Vertical tab
  \item \verb+\'+: Apostrophe atau single quote
  \item \verb+\"+: Double quote
  \item \verb+\\+:	Backslash (\verb+\+).
  \item \verb+\XXX+: Karakter dengan pengkodean Latin-1 dengan tiga digit octal antara 0 and 377. (misal, \verb+\+251 adalah simbol hak cipta).
  \item \verb+\xXX+: seperti di atas, tetapi hexadecimal (2 digit).
  \item \verb+\uXXXX+: Karakter \textit{Unicode} dengan 3 digit karakter hexadecimal.
\end{itemize}

Backslash sendiri sering digunakan sebagai \textit{escape character}, misalnya ``NaN sering disebut juga sebagai \verb+\+''Not a Number\verb+\+``''.

\subsection{Struktur Data dan Representasi JSON}

\index{JSON}JSON (\textit{JavaScript Object Notation}) adalah subset dari JavaScript dan merupakan struktur data native di JavaScript. Bentuk dari representasi struktur data JSON adalah sebagai berikut\footnote{\url{http://en.wikipedia.org/wiki/JSON} dengan sedikit perubahan}:

\lstset{language=JavaScript,caption=Representasi \textit{JSON}}
\lstinputlisting{src/bab-02/json.js}

Dari representasi di atas, kita bisa membaca:
\begin{itemize}
  \item Nilai data ``firstname'' adalah ``John''
  \item Data ``address'' terdiri atas sub data ``streetAddress'', ``city'', ``state'', dan ``postalCode'' yang masing-masing mempunyai nilai data sendiri-sendiri.
  \item dan seterusnya
\end{itemize}

\subsection{Aliran Kendali}

\index{Aliran kendali}Alur program dikendalikan melalui pernyataan-pernyataan untuk aliran kendali. Ada beberapa pernyataan aliran kendali yang akan dibahas.

\subsubsection{Pernyataan Kondisi \textit{if .. else if .. else}}

\index{if}Pernyataan ini digunakan untuk mengerjakan atau tidak mengerjakan suatu bagian atau blok program berdasarkan hasil evaluasi kondisi tertentu.

\lstset{language=JavaScript,caption=Pernyataan \textit{if .. else if .. else}}
\lstinputlisting{src/bab-02/if.js}

\subsubsection{Pernyataan \textit{switch}}

\index{switch}Pernyataan ini digunakan untuk mengevaluasi suatu ekspresi dan membandingkan sama atau tidaknya dengan suatu label tertentu di dalam struktur pernyataan switch, serta mengeksekusi perintah-perintah sesuai dengan label yang cocok.

\lstset{language=JavaScript,caption=Pernyataan \textit{Contoh penggunaan ``switch''}}
\lstinputlisting{src/bab-02/switch.js}

\subsubsection{\textit{Looping}}

\index{Looping}Looping atau sering juga disebut ``kalang'' adalah konstruksi program yang digunakan untuk melakukan suatu blok perintah secara berulang-ulang.

\textbf{for}

\lstset{language=JavaScript,caption=Pernyataan \textit{for}}
\lstinputlisting{src/bab-02/for.js}

Pernyataan ``for'' juga bisa digunakan untuk mengakses data yang tersimpam dalam struktur data JavaScript (JSON).

\lstset{language=JavaScript,caption=Pernyataan \textit{for .. in}}
\lstinputlisting{src/bab-02/forIn.js}

\textbf{do .. while}

Pernyataan ini digunakan untuk mengerjakan suatu blok program selama suatu kondisi bernilai benar dengan jumlah minimal pengerjaan sebanyak 1 kali.

\lstset{language=JavaScript,caption=Pernyataan \textit{do .. while}}
\lstinputlisting{src/bab-02/doWhile.js}

\textbf{while}

Seperti \textit{do .. while}, pernyataan ini digunakan untuk mengerjakan suatu blok program secara berulang-ulang selama kondisi bernilai benar. Meskipun demikian, bisa saja blok program tersebut tidak pernah dikerjakan jika pada saat awal expresi dievaluasi sudah bernilai \textit{false}.

\lstset{language=JavaScript,caption=Pernyataan \textit{while}}
\lstinputlisting{src/bab-02/while.js}

\textbf{label, break, dan continue}

Bagian ini digunakan dalam \textit{looping} dan \textit{switch}.
\begin{itemize}
  \item \textit{label} digunakan untuk memberi pengenal pada suatu lokasi program sehingga bisa direferensi oleh \textit{break} maupun \textit{continue} (jika dikehendaki). 
  \item \textit{break} digunakan untuk menghentikan eksekusi dan meneruskan alur program ke pernyataan setelah \textit{looping} atau \textit{switch}
  \item \textit{continue} digunakan untuk meneruskan eksekusi ke iterasi atau ke kondisi switch berikutnya.
\end{itemize}

\lstset{language=JavaScript,caption=Pernyataan \textit{break} dan \textit{continue}}
\lstinputlisting{src/bab-02/breakContinue.js}

\lstset{language=JavaScript,caption=Pernyataan \textit{break} dengan label}
\lstinputlisting{src/bab-02/breakWithLabel.js}

\subsection{Penanganan Error}

\index{Penanganan error}JavaScript mendukung pernyataan \textit{try .. catch .. finally} serta \textit{throw} untuk menangani error. Meskipun demikian, banyak hal yang tidak sesuai dengan konstruksi ini karena sifat JavaScript yang \textit{asynchronous}. Untuk kasus asynchrous, pemrogram lebih disarankan menggunakan \textit{function callback}.

\lstset{language=JavaScript,caption=Pernyataan \textit{try catch finally}}
\lstinputlisting{src/bab-02/try.js}

Jika diperlukan, kita bisa mendefinisikan sendiri error dengan menggunakan pernyataan \textit{throw}.

\lstset{language=JavaScript,caption=Pernyataan \textit{try catch throw}}
\lstinputlisting{src/bab-02/throw.js}

\chapter{Paradigma Pemrograman di JavaScript}

\section{Pemrograman Fungsional}

Pemrograman fungsional, atau sering disebut \textit{functional programming}, selama ini lebih sering dibicarakan di level para akademisi. Meskipun demikian, saat ini terdapat kecenderungan paradigma ini semakin banyak digunakan di industri. Contoh nyata dari implementasi paradigma ini di industri antara lain adalah Scala (\url{http://www.scala-lang.org}), OCaml (\url{http://www.ocaml.org}), Haskell (\url{http://www.haskell.org}), Microsoft F\# (\url{http://fsharp.org}), dan lain-lain. Dalam konteks paradigma pemrograman, peranti lunak yang dibangun menggunakan pendekatan paradigma ini akan terdiri atas berbagai fungsi yang mirip dengan fungsi matematis. Fungsi matematis tersebut di-evaluasi dengan penekanan pada penghindaran \textit{state} serta \textit{mutable data}. Bandingkan dengan paradigma pemrograman prosedural yang menekankan pada \textit{immutable data} dan definisi berbagai prosedur dan fungsi untuk mengubah \textit{state} serta data.

JavaScript bukan merupakan bahasa pemrograman fungsional yang murni, tetapi ada banyak fitur dari pemrograman fungsional yang terdapat dalam JavaScript. Dalam hal ini, JavaScript banyak dipengaruhi oleh bahasa pemrograman Scheme (\url{http://www.schemers.org/}). Bab ini akan membahas beberapa fitur pemrograman fungsional di JavaScript. Pembahasan ini didasari pembahasan di bab sebelumnya tentang Fungsi di JavaScript.

\subsection{Ekspresi Lambda}

\index{Ekspresi Lambda}Ekspresi lambda / \textit{lambda expression} merupakan hasil karya dari ALonzo Church sekitar tahun 1930-an. Aplikasi dari konsep ini di dalam pemrograman adalah penggunaan fungsi sebagai parameter untuk suatu fungsi. Dalam pemrograman, \textit{lambda function} sering juga disebut sebagai fungsi anonimus (fungsi yang dipanggil/dieksekusi tanpa ditautkan (\textit{bound}) ke suatu \textit{identifier}). Berikut adalah implementasi dari konsep ini di JavaSCript:

\lstset{language=JavaScript,caption=Ekspresi Lambda di JavaScript}
\begin{lstlisting}
// Diambil dari 
// http://stackoverflow.com/questions/3865335/what-is-a-lambda-language
// dengan beberapa perubahan

function applyOperation(a, b, operation) {
	return operation(a, b);
}

function add(a, b) {
	return a+b;
}

function subtract(a, b) {
	return a-b;
}

console.log('1,2, add: ' + applyOperation(1,2, add));
console.log('43,21, subtract: ' + applyOperation(43,21, subtract));

console.log('4^3: ' + applyOperation(4, 3, function(a,b) {return Math.pow(a, b)}))

// hasil:
// 1,2, add: 3
// 43,21, subtract: 22
// 4^3: 64
\end{lstlisting}

\subsection{Higher-order Function}

\index{Higher-order Function}\textit{Higher-order function} (sering disebut juga sebagai \textit{functor} adalah suatu fungsi yang setidak-tidaknya menggunakan satu atau lebih fungsi lain sebagai parameter dari fungsi, atau menghasilkan fungsi sebagai nilai kembalian. 

\lstset{language=JavaScript,caption=Higher-order Function di JavaScript}
\begin{lstlisting}
function forEach(array, action) {
	for (var i = 0; i < array.length; i++ )
		action(array[i]);
}

function print(word) {
	console.log(word);
}

function makeUpperCase(word) {
	console.log(word.toUpperCase());
}

forEach(["satu", "dua", "tiga"], print);
forEach(["satu", "dua", "tiga"], makeUpperCase);

// hasil:
//satu
//dua
//tiga
//SATU
//DUA
//TIGA
\end{lstlisting}

\subsection{Closure}

\index{Closure}Suatu \textit{closure} merupakan definisi suatu fungsi bersama-sama dengan lingkungannya. Lingkungan tersebut terdiri atas fungsi internal serta berbagai variabel lokal yang masih tetap tersedia saat fungsi utama / closure tersebut selesai dieksekusi. 

\lstset{language=JavaScript,caption=Closure di JavaScript}
\begin{lstlisting}
// Diambil dengan sedikit perubahan dari:
// https://developer.mozilla.org/en-US/docs/JavaScript/Guide/Closures
function makeAdder(x) {
  return function(y) {
    return x + y;
  };
}
 
var add5 = makeAdder(5);
var add10 = makeAdder(10);
 
console.log(add5(2));  // 7
console.log(add10(2)); // 12
\end{lstlisting}

\subsection{Currying}

\index{Currying}\textit{Currying} memungkinkan pemrogram untuk membuat suatu fungsi dengan cara menggunakan fungsi yang sudah tersedia secara parsial, artinya tidak perlu menggunakan semua argumen dari fungsi yang sudah tersedia tersebut.

\lstset{language=JavaScript,caption=Currying di JavaScript}
\begin{lstlisting}
// Diambil dari:
// http://javascriptweblog.wordpress.com/2010/04/05/
// 		curry-cooking-up-tastier-functions/
// dengan sedikit perubahan

function toArray(fromEnum) {
    return Array.prototype.slice.call(fromEnum);
}

Function.prototype.curry = function() {
    if (arguments.length<1) {
        return this; //nothing to curry with - return function
    }
    var __method = this;
    var args = toArray(arguments);
    return function() {
        return __method.apply(this, args.concat(toArray(arguments)));
    }
}

var add = function(a,b) {
    return a + b;
}

//create function that returns 10 + argument
var addTen = add.curry(10); 
console.log(addTen(20)); //30
\end{lstlisting}


\section{Pemrograman Berorientasi Obyek}

\subsection{Pengertian}

\index{PBO}Pemrograman Berorientasi Obyek (selanjutnya akan disingkat PBO) adalah suatu paradigma pemrograman yang memandang bahwa pemecahan masalah pemrograman akan dilakukan melalui definisi berbagai kelas kemudian membuat berbagai obyek berdasarkan kelas yng dibuat tersebut dan setelah itu mendefinisikan interaksi antar obyek tersebut dalam memecahkan masalah pemrograman. Obyek bisa saling berinteraksi karena setiap obyek mempunyai properti (sifat / karakteristik) dan \textit{method} untuk mengerjakan suatu pekerjaan tertentu. Jadi, bisa dikatakan bahwa paradigma ini menggunakan cara pandang yang manusiawi dalam penyelesaian masalah.

Dengan demikian, inti dari PBO sebenarnya terletak pada kemampuan untuk mengabstraksikan berbagai obyek ke dalam kelas (yang terdiri atas properti serta method). Paradigma PBO biasanya juga mencakup \textit{inheritance} atau pewarisan (sehingga terbentuk skema yang terdiri atas \textit{superclass} dan \textit{subclass}). Ciri lainnya adalah \textit{polymorphism} dan \textit{encapsulation} / pengkapsulan.

JavaScript adalah bahasa pemrograman yang mendukung PBO dan merupakan implementasi dari ECMAScript. Implementasi PBO di JavaScript adalah \textit{prototype-based programming} yang merupakan salah satu subset dari PBO. Pada \textit{prototype-based programming}, kelas / \textit{class} tidak ada. Pewarisan diimplementasikan melalui \textit{prototype}.

\subsection{Definisi Obyek}

\index{Obyek}Definisi obyek dilakukan dengan menggunakan definisi \textit{function}, sementara \textit{this} digunakan di dalam definisi untuk menunjukkan ke obyek tersebut. Sementara itu, Kelas.prototype.namaMethod digunakan untuk mendefinisikan method dengan nama method namaMethod pada kelas Kelas. Perhatikan contoh pada listing berikut.

\lstset{language=JavaScript,caption=Definisi obyek di JavaScript}
\begin{lstlisting}
var url = require('url');

// Definisi obyek
function Halaman(alamatUrl) {
  this.url = alamatUrl;
  console.log("Mengakses alamat " + alamatUrl);
}

Halaman.prototype.getDomainName = function() {
  return url.parse(this.url, true).host; 
}
// sampai disini definisi obyek
// Halaman.prototype.getDomainName => menetapkan method getDomainName
// untuk obyek

var halSatu = new Halaman("http://nodejs.org/api/http.html");
var halDua = new Halaman("http://bpdp.name/login?fromHome");

console.log("Alamat URL yang diakses oleh halSatu = " + halSatu.url);
console.log("Alamat URL yang diakses oleh halDua = " + halDua.url);

console.log("Nama domain halDua = " + halDua.getDomainName());

// hasil:
//Mengakses alamat http://nodejs.org/api/http.html
//Mengakses alamat http://bpdp.name/login?fromHome
//Alamat URL yang diakses oleh halSatu = http://nodejs.org/api/http.html
//Alamat URL yang diakses oleh halDua = http://bpdp.name/login?fromHome
//Nama domain halDua = bpdp.name
\end{lstlisting}

\subsection{\textit{Inheritance} / Pewarisan}

\index{Pewarisan}Pewarisan di JavaScript bisa dicapai menggunakan \textit{prototype}. Listing program berikut memperlihatkan bagaimana pewarisan diimplementasikan di JavaScript.

\lstset{language=JavaScript,caption=Pewarisan di PBO JavaScript}
\begin{lstlisting}
// Definisi obyek
function Kelas(param) {
  this.property1 = new String(param);
}

Kelas.prototype.methodSatu = function() {
  return this.property1; 
}

var kelasSatu = new Kelas("ini parameter 1 dari kelas 1");

console.log("Property 1 dari kelasSatu = " + kelasSatu.property1);
console.log("Property 1 dari kelasSatu, diambil dari method  = " + kelasSatu.methodSatu());

// Definisi inheritance:
// SubKelas merupakan anak dari Kelas yang didefinisikan
// di atas.

SubKelas.prototype = new Kelas();
SubKelas.prototype.constructor = SubKelas;

function SubKelas(param) {
  this.property1 = new String(param);
}

// method overriding
SubKelas.prototype.methodSatu = function(keHurufBesar) {
  console.log("Ubah ke huruf besar? = " + keHurufBesar);
  if (keHurufBesar) {
    return this.property1.toUpperCase();
  } else {
    return this.property1.toLowerCase();
  }
}

SubKelas.prototype.methodDua = function() {
  console.log("Berada di method dua dari SubKelas");
}

// mari diuji
var subKelasSatu = new SubKelas("Parameter 1 Dari Sub Kelas 1");

console.log("Property 1 dari sub kelas 1 = " + subKelasSatu.property1);
console.log("Property 1 dari sub kelas 1, diambil dari method+param = " + 
  subKelasSatu.methodSatu(true));
console.log("Property 1 dari sub kelas 1, diambil dari method+param = " +
  subKelasSatu.methodSatu(false));

console.log(subKelasSatu.methodDua());
// hasil:
//
//Property 1 dari kelasSatu = ini parameter 1 dari kelas 1
//Property 1 dari kelasSatu, diambil dari method  = ini 
//parameter 1 dari kelas 1
//Property 1 dari sub kelas 1 = Parameter 1 Dari Sub Kelas 1
//Ubah ke huruf besar? = true
//Property 1 dari sub kelas 1, diambil dari method+param = 
//PARAMETER 1 DARI SUB KELAS 1
//Ubah ke huruf besar? = false
//Property 1 dari sub kelas 1, diambil dari method+param = 
//parameter 1 dari sub kelas 1
//Berada di method dua dari SubKelas
\end{lstlisting}

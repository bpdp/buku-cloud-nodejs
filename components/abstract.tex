\begin{abstract}
\thispagestyle{plain}
\setcounter{page}{1}
\addcontentsline{toc}{chapter}{\numberline{}Kata Pengantar}

Buku bebas ini merupakan buku yang dirancang untuk keperluan memberikan pengetahuan mendasar pengembangan aplikasi berbasis Cloud Computing, khususnya menggunakan Node.js. Pada buku ini akan dibahas penggunaan Node.js untuk mengembangkan aplikasi SaaS (\textit{Software as a Service}). Node.js merupakan software di sisi server yang dikembangkan dari \textit{engine} JavaScript V8 dari Google serta \textbf{libuv} (\url{https://github.com/joyent/libuv})\footnote{Versi sebelum 0.9.0 menggunakan \textbf{libev} dari Mark Lechmann}. 

Jika selama ini kebanyakan orang mengenal JavaScript hanya di sisi klien (browser), dengan Node.js ini, pemrogram bisa menggunakan JavaScript di sisi server. Meskipun ini bukan hal baru, tetapi paradigma pemrograman yang dibawa oleh Node.js dengan \textit{evented - asynchronous I/O} menarik dalam pengembangan aplikasi Web (selain kita hanya perlu menggunakan 1 bahasa yang sama di sisi server maupun di sisi klien). 

Untuk mengikuti materi yang ada pada buku ini, pembaca diharapkan menyiapkan peranti komputer dengan beberapa software berikut terpasang:
\begin{itemize}
  \item Sistem operasi Linux (distribusi apa saja) - lihat di \url{http://www.distrowatch.com}. Sistem operasi Linux ini bukan keharusan, anda bisa menggunakan Windows tetapi silahkan membuat penyesuaian-penyesuaian sendiri yang diperlukan. Kirimi saya \textit{pull request} jika anda menuliskan pengalaman anda di Windows!
	\item Git (untuk \textit{version control system}) - bisa diperoleh di \url{http://git-scm.com}. Saya menggunakan versi 1.8.2.2.
	\item Ruby (\url{http://www.ruby-lang.org/en/}) - diperlukan untuk menginstall dan mengeksekusi \textit{vmc}, perintah \textit{command line} untuk mengelola aplikasi Cloud di CloudFoundry. Versi Ruby yang digunakan adalah versi 2.0.0p0 (2013-02-24 revision 39474).
	\item mongoDB (basis data NOSQL) - bisa diperoleh di \url{http://www.mongodb.org}, saya menggunakan versi 2.4.4-pre-
	\item Vim (untuk mengedit source code) - bisa diperoleh di \url{http://www.vim.org}. Jika tidak terbiasa menggunakan Vim, bisa menggunakan editor teks lainnya (atau IDE), misalnya gedit (ada di GNOME), geany (\url{http://geany.org}), KATE (ada di KDE), dan lain-lain.
\end{itemize}

Software utama untuk keperluan workshop ini yaitu Node.js serta command line tools dari provider Cloud Computing (materi ini menggunakan fasilitas dari CloudFoundry), akan dibahas pada pada bab-bab tertentu. Materi akan lebih banyak berorientasi ke command line / shell sehingga para pembaca sebaiknya sudah memahami cara-cara menggunakan shell di Linux. Anda bisa menggunakan shell apa saja (bash, tcsh, zsh, ksh, dan lain-lain), saya menggunakan bash 4.2.45(2)-release.

Have fun!
\end{abstract}
